% !TEX TS-program = lualatex
% !TEX encoding = UTF-8 Unicode

\documentclass{article}
\usepackage[utf8]{inputenc}
% 页眉
\usepackage{fancyhdr}
% 为中文 / 日语提供注音
% \usepackage{luatexja-ruby}
% 代码高亮
\usepackage{minted}
% 页面布局
\usepackage[
  papersize={8.5in,11in},
  a4paper,
  bindingoffset=0.2in,
  left=1.2in,
  right=1.2in,
  top=1in,
  bottom=1in,
  footskip=.25in,
  headheight=14pt,
]{geometry}
% 预定义颜色
\usepackage[dvipsnames]{xcolor}
% 超链接
\usepackage[
  colorlinks,
  linkcolor = BrickRed,
  citecolor = Green,
  filecolor = Mulberry,
  urlcolor  = NavyBlue,
  menucolor = BrickRed,
  runcolor  = Mulberry,
]{hyperref}
% 参考文献
\usepackage[
  backend  = biber,
  style    = caspervector,
  seconds,
  utf8,
  backref,
]{biblatex}
% 页脚
\usepackage[
  perpage,
  hang,
  flushmargin,
]{footmisc}
% 多行页脚
\usepackage{footnotebackref}
% % 字体
% \usepackage{pifont}
% 图像
\usepackage{graphicx}
% % 扫描本地字体
% \usepackage{fontspec}
% 控制章节标题格式
\usepackage{sectsty}
% 详见 https://tex.stackexchange.com/questions/664/why-should-i-use-usepackaget1fontenc
\usepackage[T1]{fontenc}
% Letter Spacing
% 详见 https://tex.stackexchange.com/questions/522106/is-there-a-way-to-get-wider-spacing-between-the-letters
\usepackage{soul} 
% 控制章节标题格式
\usepackage[explicit]{titlesec}
% 控制 TOC 目录格式
\usepackage{titletoc}
% 控制列表格式
\usepackage[shortlabels]{enumitem}
% 彩色框 - 本模板用于定义定理,定义,例子的框图样式
% 详见 https://tex.stackexchange.com/questions/562753/how-to-make-a-colour-box-in-these-3-different-ways
\usepackage{tcolorbox}
% Pst-node 节点绘图
\usepackage{pst-node}
% Tikz 绘图
\usepackage{tikz}
% Tikz 交换图绘图
\usepackage{tikz-cd}
% pgf 函数绘图
\usepackage{pgfplots}
\usetikzlibrary{positioning}
\usetikzlibrary{arrows}

% RSFS 字体
% 详解 https://tug.org/FontCatalogue/ralphsmithsformalscript/
\usepackage{mathrsfs}
% 数学
\usepackage{amssymb}
\usepackage{mathtools}
\usepackage{amsthm}
\usepackage{amsmath}
% 划线抵消删除符号
\usepackage{cancel}

% 浮动体,提供 [H] `PUT IT HERE` 选项
\usepackage{float}
% 提供更细粒度的浮动体 Caption 控制
\usepackage{caption}
% 提供并排浮动体的子 Caption 控制(左右排列的图片使用不同的 Caption)
\usepackage{subcaption}
% 提供多列的文章排版
\usepackage{multicol}
% 提供更细粒度的表格宽度控制
\usepackage{tabularx}
% 希腊字母
\usepackage{upgreek}
% 修改字体 / 数学字符的大小
\usepackage{relsize}

% 注释块
\usepackage{verbatim}

% 图像资源路径
\graphicspath{{images/}}

% 重定义粗体,以使其字体统一
\renewcommand{\emph}[1]{\textbf{#1}}

% 参考文献
\addbibresource{ref.bib}

% 代码块
\renewcommand{\theFancyVerbLine}{%
  \ttfamily {%
    \oldstylenums{\arabic{FancyVerbLine}}
  }
}
\setminted{
  mathescape,
  linenos,
  breaklines,
  breakanywhere,
  autogobble,
  numbersep=5pt,
  baselinestretch=1.2,
  frame    = lines,
  framesep = 2mm,
}

% 字体
% \setmainfont{New York Small Regular}[
%   BoldFont = New York Small Bold,
%   BoldItalicFont = New York Small Bold Italic,
%   ItalicFont = New York Small Regular Italic
% ]
% \setmainfont{CMU Serif}[
%   ItalicFont = CMU Serif Italic
% ]
% \setsansfont{SF Pro Text}[
%   ItalicFont = New York Small Regular Italic
% ]

% \newcommand{\emoji}[1]{
%   {\setmainfont{Apple Color Emoji}[Renderer=Harfbuzz]{#1}}
% }

% \setmonofont[
%   ItalicFont     = JetBrains Mono Italic,
%   BoldFont       = JetBrains Mono Bold,
%   BoldItalicFont = JetBrains Mono Bold Italic,
%   Contextuals    = Alternate,
% ]{JetBrains Mono}

% \newfontfamily\lmmono{Latin Modern Mono}
% \newfontfamily\bsans{SF Pro Text Bold}
% \newfontfamily\displaysans{SF Pro Display}

% 设置所有章节标题字体为非衬线字体
\allsectionsfont{\sf}

\sodef\chapterso{}{.5em}{2em}{2em}

% 设置章节标题样式
\titleformat{\chapter}[display]
{\bfseries} %format
{\textsf{\textbf{\large\chapterso{\MakeUppercase\chaptertitlename}\hspace{1.1em}\thechapter}}} %label
{1.5em} %sep 
{\sf{\textbf{\huge#1}}\bfseries} %before-code

\titlespacing*{\chapter}{0pt}{-30pt}{40pt}

% 设置 TOC 样式
\titlecontents{chapter}
[1.5em]
{\sf}
{\textbf{\sf{Chapter \thecontentslabel. }}}
{\huge}
{\mdseries\titlerule*[0.75em]{.}\bfseries \thecontentspage} 
[\addvspace{.5pc}]

\titlecontents{section}
[2.5em]
{\sf}
{\thecontentslabel. }
{\huge}
{\mdseries\titlerule*[0.75em]{.}\bfseries \thecontentspage} 
[\addvspace{.5pc}]

\titlecontents{subsection}
[3.5em]
{\sf}
{\thecontentslabel. }
{\huge}
{\mdseries\titlerule*[0.75em]{.}\bfseries \thecontentspage} 
[\addvspace{.5pc}]

\titlecontents{subsubsection}
[3.5em]
{\sf}
{\thecontentslabel. }
{\huge}
{\mdseries\titlerule*[0.75em]{.}\bfseries \thecontentspage} 
[\addvspace{.5pc}]

% 链接样式
\urlstyle{same}
\let\oldurl\url
\renewcommand{\url}[1]{%
{\lmmono\oldurl{#1}}
}

% 彩色定理 / 例子 / 定义环境
\definecolor{theorembg}{HTML}{F2F2F9}
\definecolor{theoremfr}{HTML}{00007B}
\definecolor{examplebg}{HTML}{F8E5D5}
\definecolor{examplefr}{HTML}{FA893F}
\definecolor{exampleti}{HTML}{2A7F7F}
\definecolor{definitbg}{HTML}{DEFFD4}
\definecolor{definitfr}{HTML}{60AD49}
\theoremstyle{definition}
\newtheorem{definition}{Definition}
\newtheorem{example}{Example}
\newtheorem{theorem}{Theorem}
\newtheorem{corollary}{Corollary}[theorem]
\newtheorem{lemma}[theorem]{Lemma}
\newtheorem*{remark}{Remark}
\newtheorem{exercise}{Exercise}
\newtheorem{problem}{Problem}
\newtheorem{proposition}{Proposition}
\newtheorem{question}{Question}
\newtheorem{conjecture}{Conjecture}
\newtheorem{assumption}{Assumption}
\newtheorem{notation}{Notation}
\newtheorem{claim}{Claim}
\newtheorem{observation}{Observation}
\newtheorem{fact}{Fact}
\newtheorem{solution}{Solution}
\newtheorem{convention}{Convention}


\tcbuselibrary{theorems,skins,hooks}

\newtcbtheorem[number within=section]{Theorem}{Theorem}
{%
   enhanced
  ,colback = theorembg
  ,frame hidden
  ,boxrule = 0sp
  ,borderline west = {2pt}{0pt}{theoremfr}
  ,sharp corners
  ,detach title
  ,before upper = \tcbtitle\par\smallskip
  ,coltitle = theoremfr
  ,fonttitle = \bfseries\sffamily
  ,description font = \mdseries
  ,terminator sign none
  ,separator sign none
}
{th}

\newtcbtheorem[number within=section]{Example}{Example}
{%
   enhanced
  ,colback = examplebg
  ,frame hidden
  ,boxrule = 0sp
  ,borderline west = {2pt}{0pt}{examplefr}
  ,sharp corners
  ,detach title
  ,before upper = \tcbtitle\par\smallskip
  ,coltitle = examplefr
  ,fonttitle = \bfseries\sffamily
  ,description font = \mdseries
  ,terminator sign none
  ,separator sign none
}
{ex}


\newtcbtheorem[number within=section]{Definition}{Definition}
{%
   enhanced
  ,colback = definitbg
  ,frame hidden
  ,boxrule = 0sp
  ,borderline west = {2pt}{0pt}{definitfr}
  ,sharp corners
  ,detach title
  ,before upper = \tcbtitle\par\smallskip
  ,coltitle = definitfr
  ,fonttitle = \bfseries\sffamily
  ,description font = \mdseries
  ,terminator sign none
  ,separator sign none
}
{def}

%% Backref
\makeatletter
\LetLtxMacro{\BHFN@Old@footnotemark}{\@footnotemark}

\renewcommand*{\@footnotemark}{%
    \refstepcounter{BackrefHyperFootnoteCounter}%
    \xdef\BackrefFootnoteTag{bhfn:\theBackrefHyperFootnoteCounter}%
    \label{\BackrefFootnoteTag}%
    \BHFN@Old@footnotemark
}
\makeatother


% 页眉
\fancyhf{}
\renewcommand{\headrulewidth}{0pt}
% \renewcommand\headrule{\makebox[\textwidth][r]{\rule{0.4\headwidth}{\headrulewidth}}}
\renewcommand{\sectionmark}[1]{\markright{#1}}
\rhead{\sf{\rightmark} \hspace{.8em} \textbf{\textsf{\thepage}}}
\pagestyle{fancy}


% 页脚
\fancyfoot[C]{\thepage}


\date{\sf{Jan 30, 2023}}
\title{\sf{MATH 541 L1 Notes}}

\begin{document}

% 标题
\maketitle

% 目录
% {
%   \hypersetup{linkcolor=black}
%   \tableofcontents
% }

% --- 预定义快捷方式 ---

\newcommand{\ra}{\rightarrow}
\newcommand{\la}{\leftarrow}
\newcommand{\suchthat}{\textnormal{ such that }}
\newcommand{\for}{\textnormal{ for }}
\newcommand{\where}{\textnormal{ where }}
\newcommand{\by}{\textnormal{ by }}
% \newcommand{\and}{\textnormal{ and }}

\newcommand{\R}{\mathbb{R}}
\newcommand{\Z}{\mathbb{Z}}
\newcommand{\N}{\mathbb{N}}
\newcommand{\Q}{\mathbb{Q}}
\newcommand{\M}{\mathbb{M}}
\newcommand{\C}{\mathbb{C}}
\newcommand{\F}{\mathbb{F}}

\renewcommand{\P}{\mathcal{P}}
\newcommand{\T}{\mathcal{T}}
\newcommand{\D}{\mathcal{D}}
\newcommand{\U}{\mathcal{U}}
\newcommand{\V}{\mathcal{V}}
\newcommand{\card}{\textnormal{card}}

\newcommand{\Mmn}{\M_{m,n}}
\newcommand{\FF}[2]{\F^{#1 \times #2}}
\newcommand{\RR}[2]{\R^{#1 \times #2}}
\newcommand{\CC}[2]{\C^{#1 \times #2}}
\newcommand{\seq}[1]{\{#1_i\}_{i=1}^{\infty}}
\newcommand{\trace}{\operatorname{tr}}
\newcommand{\rsa}{\rightsquigarrow}
\newcommand{\rank}{\operatorname{rank}}
\newcommand{\diag}{\operatorname{diag}}
\newcommand{\spann}[1]{\operatorname{span}(#1)}
\newcommand{\proj}{\operatorname{proj}}
\newcommand{\vol}{\operatorname{vol}}
\newcommand{\cis}{\operatorname{cis}}
\newcommand{\Sym}{\operatorname{Sym}}
% partial qed
\newcommand{\qedp}{\hfill $\blacksquare$}
\newcommand{\qedf}{\hfill $\square$}
\newcommand{\oneb}{\mathbf{1}}
\newcommand{\zerob}{\mathbf{0}}
\newcommand\defeq{\stackrel{\mathclap{\normalfont\scriptsize\mbox{def}}}{=}}

\newcommand{\mat}[1]{\begin{bmatrix} #1 \end{bmatrix}}
\newcommand{\vmat}[1]{\begin{vmatrix} #1 \end{vmatrix}}
\newcommand{\pmat}[1]{\begin{pmatrix} #1 \end{pmatrix}}
\DeclarePairedDelimiter\ceil{\lceil}{\rceil}
\DeclarePairedDelimiter\floor{\lfloor}{\rfloor}

\renewcommand\labelenumi{(\theenumi)}
\renewcommand\qedsymbol{$\square$}

\let\oldemptyset\emptyset
\let\emptyset\varnothing

\newcommand{\citem}[1]{\setcounter{enumi}{#1}\item}

% --- 正文 ---

We will cover basics of \emph{groups}, \emph{rings}, and \emph{modules}. There are all \emph{sets} with additional structures.

\begin{Example}{}{}
  $\mathbb{R}$ is a ring (a field). A vector space over $\mathbb{R}$ is a module.
\end{Example}

\section{Recap of Sets}

$A, B$ are sets, $f: A \to B$ is a function.

\begin{Definition}{Injection}{injection}
  $f$ is an \emph{injection} if $f(a) = f(b) \implies a = b$. 
\end{Definition}

\begin{Example}{}{}
  $f: \mathbb{R} \to \mathbb{R}, x \mapsto x^2$ is not an injection. $f(2) = f(-2), 2 \neq -2$ 
\end{Example}

\begin{Definition}{Surjection}{}
  $f$ is a \emph{surjection} if $\forall b \in B, \exists a \in A, \text{s.t. } f(a) = b$
\end{Definition}

\begin{Definition}{Bijection}{}
  $f$ is a \emph{bijection} if $f$ is both a \emph{surjection} and an \emph{injection}. 
  $f$ is bijective $\iff$ $f$ has an unique inverse function $f^{-1}$. 

  $$
  f^{-1}(f(a)) = a \; \forall a \in A, \quad f(f^{-1}(b)) = b \; \forall b \in B
  $$
\end{Definition}

\subsection{Products of Sets}

\begin{Definition}{Products of Sets}{}
  $A, B$ are sets, $A \times B$ is the set of all ordered pairs $(a, b)$ where $a \in A, b \in B$. $A \times B = \{(a, b) | a \in A, b \in B\}$
\end{Definition}

\begin{Example}{}{}
  $\mathbb{R}^{2} = \mathbb{R} \times \mathbb{R}$ 
\end{Example}

\section{Binary Operation}

\begin{Definition}{Binary Operation}{}
  Binary Operation on a set $X$ is a function $*$
  \[
  *: X \times X \to X, (x, y) \mapsto  x + y
  \]
\end{Definition}

\begin{Example}{}{}
  $X = \mathbb{Z}$, $* = +$ is a binary operation on $\mathbb{R}$, $3 + 5 = 8$
\end{Example}

\begin{Example}{}{}
  Consider the set $[n] = \{1, 2, \cdots, n \}$, 
  \[
  \text{Aut}([n]) = \{f:[n] \rightarrow [n]~|~f \text{ is bijective}\}
  \] 
  $n = 3$, $f = (2, 1, 3) = (1, 3, 2)$ (Cycle Representation), can form
  \begin{figure}[H]
    \centering
    \begin{tikzpicture}[
      on grid,
      every node/.style={anchor=base,minimum size=10mm},
      node distance=6mm and 2.5cm,
      ]
      % First row
      \node (ab)                {\([n]\)};
      \node (Rn)  [right=of ab] {\([n]\)};
      \node (R)   [right=of Rn] {\([n]\)};
      % Arrows
      \draw [->] (ab) to node [above] {\(g\)} (Rn);
      \draw [->] (Rn) to node [above] {\(f\)} (R);
      \draw [->, bend right] (ab) to node [below] {\(f \circ g\)} (R);
    \end{tikzpicture}
  \end{figure}
  \[
  \begin{aligned}
    f \circ g(1) &= f(g(1)) = f(1) = 3 \\
    f \circ g(2) &= f(g(2)) = f(3) = 2 \\
    f \circ g(3) &= f(g(3)) = f(2) = 1 \\
    f \circ g &= (3, 1) = (1, 3)
  \end{aligned}
  \]
  $(\text{Aut}[n], \circ)$ forms a group.
\end{Example}

\section{Group}

\begin{Definition}{Group}{}
  A group $G$ is a \emph{set} equipped with a binary operation $*$ such that:
  \begin{itemize}
    \item Associative: $(a * b) * c = a * (b * c), \; \forall a, b, c \in G$
    \item Identity: $\exists e \in G, \; e * a = a * e = a, \; \forall a \in G$
    \item Inverse: $\forall a \in G, \; \exists a^{-1} \in G, \; a * a^{-1} = a^{-1} * a = e$
  \end{itemize}
\end{Definition}

\begin{Example}{}{}
  Check $(\text{Aut}[n], \circ)$ is a group.
  \begin{itemize}
    \item Associative: $(f \circ g) \circ h = f \circ (g \circ h)$. This is an equality of functions: $[n] \to [n]$.
    \newline i.e. $\forall x \in [n], (f \circ g) \circ h(x) = f \circ (g \circ h) (x) = f(g(h(x)))$
    \item Identity: $\exists e \in \text{Aut}[n], \; e \circ f = f \circ e = f$. i.e $\text{id}_{[n]}(x) = x, \forall x \in [n]$, namely the permutation that does nothing.
    \item Inverse: $f \in \text{Aut}([n])$ is bijective. 
    \newline i.e. $\forall f \in \text{Aut}[n], \; \exists f^{-1} \in \text{Aut}[n], \; f \circ f^{-1} = f^{-1} \circ f = \text{id}_{[n]}$. 
  \end{itemize}
\end{Example}


\begin{exercise}
  Compute $(1, 2, 3) \circ (2, 3)$ and $(2, 3) \circ (1, 2, 3)$ \\
  \[
  \begin{aligned}
    (1, 2, 3) \circ (2, 3) &= (2, 1, 3) \\
    (2, 3) \circ (1, 2, 3) &= (1, 3)
  \end{aligned}
  \]
\end{exercise}

In general, for a group $(G, *)$, $a * b \neq b*a$ (not necessarily)

\begin{Definition}{Abelian Group}{}
  If $a*b = b*a, \forall a, b \in G$, then $G$ is called \emph{abelian}, or \emph{commutative}.
\end{Definition}

\begin{Example}{}{}
  $(\mathbb{Z}, +)$ is an abelian group. \\ 
  $(\mathbb{Z}, *)$ is \emph{NOT} a group! (Inverse of $0$ does not exist) \\
  $({\pm 1}, \times )$ is an abelian group. \\
  $M_{n\times n} = \{n \times n \text{ matrices / } \mathbb{R}\}$ $(M_{n \times n}, + )$ is an abelian group. \\
  $M^{\times }_{n\times n} = \{A \in M_{n\times n} | \operatorname{det}(A) \neq 0\}$. Then $(M^{\times }_{n\times n}, \times )$ is a group. \\
  $\mathbb{R}^{n} \to \mathbb{R}^{n}$ usually not commutative.
\end{Example}

\end{document}