\input{./include/tex_template_nofont.tex}

\date{\sf{Jan 30, 2023}}
\title{\sf{MATH 541 L1 Notes}}

\begin{document}

% 标题
\maketitle

% 目录
% {
%   \hypersetup{linkcolor=black}
%   \tableofcontents
% }

\input{./include/shortcut.tex}

% --- 正文 ---

We will cover basics of \emph{groups}, \emph{rings}, and \emph{modules}. There are all \emph{sets} with additional structures.

\begin{Example}{}{}
  $\mathbb{R}$ is a ring (a field). A vector space over $\mathbb{R}$ is a module.
\end{Example}

\section{Recap of Sets}

$A, B$ are sets, $f: A \to B$ is a function.

\begin{Definition}{Injection}{injection}
  $f$ is an \emph{injection} if $f(a) = f(b) \implies a = b$. 
\end{Definition}

\begin{Example}{}{}
  $f: \mathbb{R} \to \mathbb{R}, x \mapsto x^2$ is not an injection. $f(2) = f(-2), 2 \neq -2$ 
\end{Example}

\begin{Definition}{Surjection}{}
  $f$ is a \emph{surjection} if $\forall b \in B, \exists a \in A, \text{s.t. } f(a) = b$
\end{Definition}

\begin{Definition}{Bijection}{}
  $f$ is a \emph{bijection} if $f$ is both a \emph{surjection} and an \emph{injection}. 
  $f$ is bijective $\iff$ $f$ has an unique inverse function $f^{-1}$. 

  $$
  f^{-1}(f(a)) = a \; \forall a \in A, \quad f(f^{-1}(b)) = b \; \forall b \in B
  $$
\end{Definition}

\subsection{Products of Sets}

\begin{Definition}{Products of Sets}{}
  $A, B$ are sets, $A \times B$ is the set of all ordered pairs $(a, b)$ where $a \in A, b \in B$. $A \times B = \{(a, b) | a \in A, b \in B\}$
\end{Definition}

\begin{Example}{}{}
  $\mathbb{R}^{2} = \mathbb{R} \times \mathbb{R}$ 
\end{Example}

\section{Binary Operation}

\begin{Definition}{Binary Operation}{}
  Binary Operation on a set $X$ is a function $*$
  \[
  *: X \times X \to X, (x, y) \mapsto  x + y
  \]
\end{Definition}

\begin{Example}{}{}
  $X = \mathbb{Z}$, $* = +$ is a binary operation on $\mathbb{R}$, $3 + 5 = 8$
\end{Example}

\begin{Example}{}{}
  Consider the set $[n] = \{1, 2, \cdots, n \}$, 
  \[
  \text{Aut}([n]) = \{f:[n] \rightarrow [n]~|~f \text{ is bijective}\}
  \] 
  $n = 3$, $f = (2, 1, 3) = (1, 3, 2)$ (Cycle Representation), can form
  \begin{figure}[H]
    \centering
    \begin{tikzpicture}[
      on grid,
      every node/.style={anchor=base,minimum size=10mm},
      node distance=6mm and 2.5cm,
      ]
      % First row
      \node (ab)                {\([n]\)};
      \node (Rn)  [right=of ab] {\([n]\)};
      \node (R)   [right=of Rn] {\([n]\)};
      % Arrows
      \draw [->] (ab) to node [above] {\(g\)} (Rn);
      \draw [->] (Rn) to node [above] {\(f\)} (R);
      \draw [->, bend right] (ab) to node [below] {\(f \circ g\)} (R);
    \end{tikzpicture}
  \end{figure}
  \[
  \begin{aligned}
    f \circ g(1) &= f(g(1)) = f(1) = 3 \\
    f \circ g(2) &= f(g(2)) = f(3) = 2 \\
    f \circ g(3) &= f(g(3)) = f(2) = 1 \\
    f \circ g &= (3, 1) = (1, 3)
  \end{aligned}
  \]
  $(\text{Aut}[n], \circ)$ forms a group.
\end{Example}

\section{Group}

\begin{Definition}{Group}{}
  A group $G$ is a \emph{set} equipped with a binary operation $*$ such that:
  \begin{itemize}
    \item Associative: $(a * b) * c = a * (b * c), \; \forall a, b, c \in G$
    \item Identity: $\exists e \in G, \; e * a = a * e = a, \; \forall a \in G$
    \item Inverse: $\forall a \in G, \; \exists a^{-1} \in G, \; a * a^{-1} = a^{-1} * a = e$
  \end{itemize}
\end{Definition}

\begin{Example}{}{}
  Check $(\text{Aut}[n], \circ)$ is a group.
  \begin{itemize}
    \item Associative: $(f \circ g) \circ h = f \circ (g \circ h)$. This is an equality of functions: $[n] \to [n]$.
    \newline i.e. $\forall x \in [n], (f \circ g) \circ h(x) = f \circ (g \circ h) (x) = f(g(h(x)))$
    \item Identity: $\exists e \in \text{Aut}[n], \; e \circ f = f \circ e = f$. i.e $\text{id}_{[n]}(x) = x, \forall x \in [n]$, namely the permutation that does nothing.
    \item Inverse: $f \in \text{Aut}([n])$ is bijective. 
    \newline i.e. $\forall f \in \text{Aut}[n], \; \exists f^{-1} \in \text{Aut}[n], \; f \circ f^{-1} = f^{-1} \circ f = \text{id}_{[n]}$. 
  \end{itemize}
\end{Example}


\begin{exercise}
  Compute $(1, 2, 3) \circ (2, 3)$ and $(2, 3) \circ (1, 2, 3)$ \\
  \[
  \begin{aligned}
    (1, 2, 3) \circ (2, 3) &= (2, 1, 3) \\
    (2, 3) \circ (1, 2, 3) &= (1, 3)
  \end{aligned}
  \]
\end{exercise}

In general, for a group $(G, *)$, $a * b \neq b*a$ (not necessarily)

\begin{Definition}{Abelian Group}{}
  If $a*b = b*a, \forall a, b \in G$, then $G$ is called \emph{abelian}, or \emph{commutative}.
\end{Definition}

\begin{Example}{}{}
  $(\mathbb{Z}, +)$ is an abelian group. \\ 
  $(\mathbb{Z}, *)$ is \emph{NOT} a group! (Inverse of $0$ does not exist) \\
  $({\pm 1}, \times )$ is an abelian group. \\
  $M_{n\times n} = \{n \times n \text{ matrices / } \mathbb{R}\}$ $(M_{n \times n}, + )$ is an abelian group. \\
  $M^{\times }_{n\times n} = \{A \in M_{n\times n} | \operatorname{det}(A) \neq 0\}$. Then $(M^{\times }_{n\times n}, \times )$ is a group. \\
  $\mathbb{R}^{n} \to \mathbb{R}^{n}$ usually not commutative.
\end{Example}

\end{document}