% !TEX TS-program = lualatex
% !TEX encoding = UTF-8 Unicode

\documentclass{article}
\usepackage[utf8]{inputenc}
% 页眉
\usepackage{fancyhdr}
% 为中文 / 日语提供注音
% \usepackage{luatexja-ruby}
% 代码高亮
\usepackage{minted}
% 页面布局
\usepackage[
  papersize={8.5in,11in},
  bindingoffset=0.2in,
  left=1.2in,
  right=1.2in,
  top=1in,
  bottom=1in,
  footskip=.25in,
  headheight=14pt,
]{geometry}
% 预定义颜色
\usepackage[dvipsnames]{xcolor}
% 超链接
\usepackage[
  colorlinks,
  linkcolor = BrickRed,
  citecolor = Green,
  filecolor = Mulberry,
  urlcolor  = NavyBlue,
  menucolor = BrickRed,
  runcolor  = Mulberry,
]{hyperref}
% 参考文献
\usepackage[
  backend  = biber,
  style    = caspervector,
  seconds,
  utf8,
  backref,
]{biblatex}
% 页脚
\usepackage[
  perpage,
  hang,
  flushmargin,
]{footmisc}
% 多行页脚
\usepackage{footnotebackref}
% % 字体
% \usepackage{pifont}
% 图像
\usepackage{graphicx}
% % 扫描本地字体
% \usepackage{fontspec}
% 控制章节标题格式
\usepackage{sectsty}
% 详见 https://tex.stackexchange.com/questions/664/why-should-i-use-usepackaget1fontenc
\usepackage[T1]{fontenc}
% Letter Spacing
% 详见 https://tex.stackexchange.com/questions/522106/is-there-a-way-to-get-wider-spacing-between-the-letters
\usepackage{soul} 
% 控制章节标题格式
\usepackage[explicit]{titlesec}
% 控制 TOC 目录格式
\usepackage{titletoc}
% 控制列表格式
\usepackage[shortlabels]{enumitem}
% 彩色框 - 本模板用于定义定理,定义,例子的框图样式
% 详见 https://tex.stackexchange.com/questions/562753/how-to-make-a-colour-box-in-these-3-different-ways
\usepackage{tcolorbox}
% Pst-node 节点绘图
\usepackage{pst-node}
% Tikz 绘图
\usepackage{tikz}
% Tikz 交换图绘图
\usepackage{tikz-cd}
% pgf 函数绘图
\usepackage{pgfplots}
\usetikzlibrary{positioning}
\usetikzlibrary{arrows}

% RSFS 字体
% 详解 https://tug.org/FontCatalogue/ralphsmithsformalscript/
\usepackage{mathrsfs}
% 数学
\usepackage{amssymb}
\usepackage{mathtools}
\usepackage{amsthm}
\usepackage{amsmath}
% 划线抵消删除符号
\usepackage{cancel}

% 浮动体,提供 [H] `PUT IT HERE` 选项
\usepackage{float}
% 提供更细粒度的浮动体 Caption 控制
\usepackage{caption}
% 提供并排浮动体的子 Caption 控制(左右排列的图片使用不同的 Caption)
\usepackage{subcaption}
% 提供多列的文章排版
\usepackage{multicol}
% 提供更细粒度的表格宽度控制
\usepackage{tabularx}
% 希腊字母
% \usepackage{upgreek}
% 修改字体 / 数学字符的大小
% \usepackage{relsize}

% 注释块
% \usepackage{verbatim}

% 图像资源路径
\graphicspath{{images/}}

% 重定义粗体,以使其字体统一
\renewcommand{\emph}[1]{\textbf{#1}}

% 参考文献
\addbibresource{ref.bib}

% 代码块
\renewcommand{\theFancyVerbLine}{%
  \ttfamily {%
    \oldstylenums{\arabic{FancyVerbLine}}
  }
}
\setminted{
  mathescape,
  linenos,
  breaklines,
  breakanywhere,
  autogobble,
  numbersep=5pt,
  baselinestretch=1.2,
  frame    = lines,
  framesep = 2mm,
}

% 字体
% \setmainfont{New York Small Regular}[
%   BoldFont = New York Small Bold,
%   BoldItalicFont = New York Small Bold Italic,
%   ItalicFont = New York Small Regular Italic
% ]
% \setmainfont{CMU Serif}[
%   ItalicFont = CMU Serif Italic
% ]
% \setsansfont{SF Pro Text}[
%   ItalicFont = New York Small Regular Italic
% ]

% \newcommand{\emoji}[1]{
%   {\setmainfont{Apple Color Emoji}[Renderer=Harfbuzz]{#1}}
% }

% \setmonofont[
%   ItalicFont     = JetBrains Mono Italic,
%   BoldFont       = JetBrains Mono Bold,
%   BoldItalicFont = JetBrains Mono Bold Italic,
%   Contextuals    = Alternate,
% ]{JetBrains Mono}

% \newfontfamily\lmmono{Latin Modern Mono}
% \newfontfamily\bsans{SF Pro Text Bold}
% \newfontfamily\displaysans{SF Pro Display}

% 设置所有章节标题字体为非衬线字体
\allsectionsfont{\sf}

\sodef\chapterso{}{.5em}{2em}{2em}

% 设置章节标题样式
\titleformat{\chapter}[display]
{\bfseries} %format
{\textsf{\textbf{\large\chapterso{\MakeUppercase\chaptertitlename}\hspace{1.1em}\thechapter}}} %label
{1.5em} %sep 
{\sf{\textbf{\huge#1}}\bfseries} %before-code

\titlespacing*{\chapter}{0pt}{-30pt}{40pt}

% 设置 TOC 样式
\titlecontents{chapter}
[1.5em]
{\sf}
{\textbf{\sf{Chapter \thecontentslabel. }}}
{\huge}
{\mdseries\titlerule*[0.75em]{.}\bfseries \thecontentspage} 
[\addvspace{.5pc}]

\titlecontents{section}
[2.5em]
{\sf}
{\thecontentslabel. }
{\huge}
{\mdseries\titlerule*[0.75em]{.}\bfseries \thecontentspage} 
[\addvspace{.5pc}]

\titlecontents{subsection}
[3.5em]
{\sf}
{\thecontentslabel. }
{\huge}
{\mdseries\titlerule*[0.75em]{.}\bfseries \thecontentspage} 
[\addvspace{.5pc}]

\titlecontents{subsubsection}
[3.5em]
{\sf}
{\thecontentslabel. }
{\huge}
{\mdseries\titlerule*[0.75em]{.}\bfseries \thecontentspage} 
[\addvspace{.5pc}]

% 链接样式
\urlstyle{same}
\let\oldurl\url
\renewcommand{\url}[1]{%
{\lmmono\oldurl{#1}}
}

% 彩色定理 / 例子 / 定义环境
\definecolor{theorembg}{HTML}{F2F2F9}
\definecolor{theoremfr}{HTML}{00007B}
\definecolor{examplebg}{HTML}{F8E5D5}
\definecolor{examplefr}{HTML}{FA893F}
\definecolor{exampleti}{HTML}{2A7F7F}
\definecolor{definitbg}{HTML}{DEFFD4}
\definecolor{definitfr}{HTML}{60AD49}
\theoremstyle{definition}
\newtheorem{definition}{\textbf{\textsf{Definition}}}
\newtheorem{example}{\textbf{\textsf{Example}}}
\newtheorem{theorem}{\textbf{\textsf{Theorem}}}
\newtheorem{corollary}[theorem]{\textbf{\textsf{Corollary}}}
\newtheorem{lemma}[theorem]{\textbf{\textsf{Lemma}}}
\newtheorem*{remark}{\textbf{\textsf{Remark}}}
\newtheorem{exercise}{\textbf{\textsf{Exercise}}}
\newtheorem{problem}{\textbf{\textsf{Problem}}}
\newtheorem{proposition}{\textbf{\textsf{Proposition}}}
\newtheorem{question}{\textbf{\textsf{Question}}}
\newtheorem{conjecture}{\textbf{\textsf{Conjecture}}}
\newtheorem{assumption}{\textbf{\textsf{Assumption}}}
\newtheorem{notation}{\textbf{\textsf{Notation}}}
\newtheorem{claim}{\textbf{\textsf{Claim}}}
\newtheorem{observation}{\textbf{\textsf{Observation}}}
\newtheorem{fact}{\textbf{\textsf{Fact}}}
\newtheorem{solution}{\textbf{\textsf{Solution}}}
\newtheorem{convention}{\textbf{\textsf{Convention}}}


\tcbuselibrary{theorems,skins,hooks}

\newtcbtheorem[number within=section]{Theorem}{Theorem}
{%
   enhanced
  ,colback = theorembg
  ,frame hidden
  ,boxrule = 0sp
  ,borderline west = {2pt}{0pt}{theoremfr}
  ,sharp corners
  ,detach title
  ,before upper = \tcbtitle\par\smallskip
  ,coltitle = theoremfr
  ,fonttitle = \bfseries\sffamily
  ,description font = \mdseries
  ,terminator sign none
  ,separator sign none
}
{th}

\newtcbtheorem[number within=section]{Example}{Example}
{%
   enhanced
  ,colback = examplebg
  ,frame hidden
  ,boxrule = 0sp
  ,borderline west = {2pt}{0pt}{examplefr}
  ,sharp corners
  ,detach title
  ,before upper = \tcbtitle\par\smallskip
  ,coltitle = examplefr
  ,fonttitle = \bfseries\sffamily
  ,description font = \mdseries
  ,terminator sign none
  ,separator sign none
}
{ex}


\newtcbtheorem[number within=section]{Definition}{Definition}
{%
   enhanced
  ,colback = definitbg
  ,frame hidden
  ,boxrule = 0sp
  ,borderline west = {2pt}{0pt}{definitfr}
  ,sharp corners
  ,detach title
  ,before upper = \tcbtitle\par\smallskip
  ,coltitle = definitfr
  ,fonttitle = \bfseries\sffamily
  ,description font = \mdseries
  ,terminator sign none
  ,separator sign none
}
{def}

%% Backref
\makeatletter
\LetLtxMacro{\BHFN@Old@footnotemark}{\@footnotemark}

\renewcommand*{\@footnotemark}{%
    \refstepcounter{BackrefHyperFootnoteCounter}%
    \xdef\BackrefFootnoteTag{bhfn:\theBackrefHyperFootnoteCounter}%
    \label{\BackrefFootnoteTag}%
    \BHFN@Old@footnotemark
}
\makeatother


% 页眉
\fancyhf{}
\renewcommand{\headrulewidth}{0pt}
% \renewcommand\headrule{\makebox[\textwidth][r]{\rule{0.4\headwidth}{\headrulewidth}}}
\renewcommand{\sectionmark}[1]{\markright{#1}}
\rhead{\sf{\rightmark} \hspace{.8em} \textbf{\textsf{\thepage}}}
\pagestyle{fancy}


% 页脚
\fancyfoot[C]{\thepage}
